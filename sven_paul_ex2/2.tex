\documentclass[a4paper]{article}
\usepackage[top=1in, bottom=1.25in, left=1.25in, right=1.25in]{geometry}


\usepackage{amsmath}
\usepackage{amssymb}
\usepackage{graphicx}
\usepackage[utf8]{inputenc}
 
\usepackage{amsthm}
\usepackage{enumitem}   
\usepackage{listings}

\lstset{
  breaklines=true,
  numbers=left,
  language=Python
}

\begin{document}

%% Title, authors and addresses

\title{Brain Inspired Computing - Problem Set 2}

\author{Sven Bordukat, Paul Meehan}

\date{\today}

\maketitle

\section*{Exercise 1}
\begin{itemize}
    \item[a)]
    The capacitance of a spherical capacitor is
    \begin{align*}
        C&=4\pi\epsilon_0\epsilon_r\frac{R_2R_1}{R_2-R_1}
    \end{align*}
    In our case we have $R_1 = 10\mu m$ and $R_2 = 9.995\mu m$.
    This gives us a capacitance of
    \begin{align*}
        C&=4\pi\epsilon_0*3*\frac{10* 9.995}{10-9.995}\mu m\\
        &=6.67 pF
        C_{specific} &= \frac{C}{A}\\
        &= \frac{6.67pF}{4\pi r^2}\\
        &= \frac{6.67pF}{4\pi (10 \mu m)^2}\\
        &= 0.531 \mu F/cm^2
    \end{align*}

    \item[b)]
    As per the definition of capacity, a shift of $U=10mV$ is achieved when
    a charge of $Q=U*C=10mV*6.67pF=6.67*10^{-14}=4.16*10^5e$ is moved.
    As a Na\textsuperscript{+} ion has a charge of $1e$, this results in
    $4.16*10^5$ Na\textsuperscript{+} ions being moved. The volume of the cell
    is $4189\mu m^3$, which means that the total number of ions inside the cell
    is about $4189*3*10^7=1.25*10^{11}$. Compared to the total
    number of ions on the inside, the shift of $10mV$ moves about $3.3*10^{-4}\%$
    of the ions.
    \item[c)]

\end{itemize}<++>

\section*{Exercise 2}
\end{document}
