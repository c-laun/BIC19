\documentclass[a4paper]{article}
\usepackage[top=1in, bottom=1.25in, left=1.25in, right=1.25in]{geometry}


\usepackage{amsmath}
\usepackage{amssymb}
\usepackage{graphicx}
\usepackage[utf8]{inputenc}
 
\usepackage{amsthm}
\usepackage{enumitem}   
\usepackage{listings}

\lstset{
  breaklines=true,
  numbers=left,
  language=Python
}

\begin{document}

%% Title, authors and addresses

\title{Brain Inspired Computing - Problem Set 1}

\author{Sven Bordukat, Paul Meehan}

\date{\today}

\maketitle

\section*{Exercise 1}
\begin{itemize}
    \item[a)]
    Let $A:=F_u+G_w$, $B:=F_uG_w-F_wG_u$ and $C:=(F_u+G_w)^2-4(F_uG_w-F_wG_u)$.
    Assume $A>0$. From the eigenvalue equation
    \begin{align*}
        \lambda_{\pm} &= \frac{1}{2}(F_u+G_w\pm\sqrt{(F_u+G_w)^2-4(F_uG_w-F_wG_u)})\\
                             &= \frac{1}{2}(A \pm \sqrt{A^2-4B})
        \label{eq:eigen}
    \end{align*}
    we derive that if $C\geq0$, $\lambda_+>0$ as well. If $C<0$, $\lambda_{\pm}$ can be
    expressed through $\lambda_{\pm}=r\pm i*\omega$ with $r=A/2>0$.
    However, any positive $\lambda$ leads to growth, which means that the model is
    not stable under these conditions. In the case of $C>0$, we have a saddle or
    an unstable situation, depending on whether $\lambda_-$ is positive or negative;
    in the case of $C<0$ we have a growing spiral.
    Therefore, $A<0$ must hold true.
    In the case of $C<0$, this is enough, as any solution that satisfies $A<0$
    will lead to $\lambda_{\pm}=r\pm i*\omega$ with $r<0$.
    However, for $C>0$, we need to make sure that $\lambda_+<0$ in any case.
    To satisfy this condition, $\sqrt{C}<abs(A)=\sqrt{A^2}$ must hold true.
    Expansion of $C$ leads to $\sqrt{A^2-4B}<\sqrt{A^2}$. From this we can easily
    see that $F_uG_w-F_wG_u=B>0$ must hold true.

\item[b)]
The fixpoints are $(-\frac{3}{2}, -\frac{3}{8})$ and
$(0,\frac{15}{8})$, respectively. For the case of $I=0$ we can simply calculate
the conditions given above. We have
\begin{align*}
    F_u &= 1-u^2\\
    F_w &= -1\\
    G_u &= \epsilon*b\\
    G_w &= -\epsilon*w\\
\end{align*}
which gives us
\begin{align*}
    F_u+G_w &= 1-u^2-\epsilon*w\\
    &= 1-\frac{9}{4}+0.1*\frac{3}{8}\\
    &= \frac{8-18+0.3}{8}\\
    &= -\frac{9.7}{8}<0
\end{align*}
and
\begin{align*}
    F_u*G_w-F_w*G_u &= (1-u^2)*(-\epsilon*w) - (-1*\epsilon*b)\\
    &= \epsilon(-\frac{5}{8}*\frac{3}{8}+\frac{3}{2})\\
    &= \epsilon(\frac{3}{2}-\frac{15}{64}) > 0\\
\end{align*}
Therefore the fixpoint for $I=0$ is stable.\\
The nullcline of $w$ is given through $G=0\Leftrightarrow w=a+b*u$.
For $I=15/8$ lets look at a point on the nullcline of $w$ at coordinates $(u_0+\delta u, a+b*(u_0+\delta u))$ with $\delta u>0, \delta u \ll 1$.
As $u_0=0$, we get
\begin{align*}
    F&=\delta u - \frac{{\delta u}^3}{3}-\frac{15}{8}+\delta u *\frac{3}{2}+\frac{15}{8}\\
    &\approx \frac{5}{2} \delta u > 0\\
\end{align*}
This means that the arrows point away from the fixpoint (at least on the nullcline)
and the fixpoint is therefore not stable.
\end{itemize}

\section*{Exercise 2}
\end{document}
